\documentclass{esannV2}
\usepackage[dvips]{graphicx}
\usepackage[latin1]{inputenc}
\usepackage{amssymb,amsmath,array}
\usepackage{float}
\newcolumntype{?}{!{\vrule width 1pt}}

\usepackage{color, colortbl}
\definecolor{LRed}{rgb}{.7,.8,.8}
\definecolor{LightCyan}{rgb}{0.88,.8,.8}
\definecolor{w-color}{rgb}{1,1,1}

\usepackage{enumitem}
%***********************************************************************
% !!!! IMPORTANT NOTICE ON TEXT MARGINS !!!!!
%***********************************************************************
%
% Please avoid using DVI2PDF or PS2PDF converters: some undesired
% shifting/scaling may occur when using these programs
% It is strongly recommended to use the DVIPS converters, and to submit
% PS file. You may submit a PDF file if and only if you use ADOBE ACROBAT
% to convert your PS file to PDF.
%
% Check that you have set the paper size to A4 (and NOT to letter) in your
% dvi2ps converter, in Adobe Acrobat if you use it, and in any printer driver
% that you could use.  You also have to disable the 'scale to fit paper' option
% of your printer driver.
%
% In any case, please check carefully that the final size of the top and
% bottom margins is 5.2 cm and of the left and right margins is 4.4 cm.
% It is your responsibility to verify this important requirement.  If these margin requirements and not fulfilled at the end of your file generation process, please use the following commands to correct them.  Otherwise, please do not modify these commands.
%
\voffset 0 cm \hoffset 0 cm \addtolength{\textwidth}{0cm}
\addtolength{\textheight}{0cm}\addtolength{\leftmargin}{0cm}

%***********************************************************************
% !!!! USE OF THE esannV2 LaTeX STYLE FILE !!!!!
%***********************************************************************
%
% Some commands are inserted in the following .tex example file.  Therefore to
% set up your ESANN submission, please use this file and modify it to insert
% your text, rather than staring from a blank .tex file.  In this way, you will
% have the commands inserted in the right place.

\begin{document}
%style file for ESANN manuscripts
\title{The Conjunctive Disjunctive Node Kernel}

%***********************************************************************
% AUTHORS INFORMATION AREA
%***********************************************************************
\author{Dinh Tran Van$^1$, Alessandro Sperduti$^1$ and Fabrizio Costa$^2$
%
% Optional short acknowledgment: remove next line if non-needed
%\thanks{This is an optional funding source acknowledgement.}
%
% DO NOT MODIFY THE FOLLOWING '\vspace' ARGUMENT
\vspace{.3cm}\\
%
% Addresses and institutions (remove "1- " in case of a single institution)
1- Department of Mathematics, Padova University\\
Trieste, 63, 35121 Padova, Italy
%
% Remove the next three lines in case of a single institution
\vspace{.1cm}\\
2- Bioinformatics Group, Department of Computer Science, Freiburg University \\
Georges-Kohler-Allee 106, 79110 Freiburg, Germany\\
}
%***********************************************************************
% END OF AUTHORS INFORMATION AREA
%***********************************************************************

\maketitle

\begin{abstract} Gene-disease associations are inferred on the basis of
similarities between genes. Biological relationships that are exploited to
define similarities range from interacting proteins, proteins that participate
in pathways and gene expression profiles. Though graph kernel methods have
become a prominent approach for association prediction, most solutions are
based on a notion of information diffusion that does not capture the
specificity of different network parts. Here we propose a graph kernel method that
explicitly models the configuration of each gene's context. An empirical
evaluation on several biological databases show that our proposal is
competitive w.r.t. state-of-the-art kernel approaches.

\end{abstract}

\section{Introduction and Related Work} \label{introduction} 

Predictive systems for gene-disease associations are often based on a notion
of similarity between genes. A common strategy is to encode relations between
genes as a network and use graph based techniques to make useful inferences.
In the last decades, a number of graph kernel methods have been proposed that
directly exploit transitive properties in biological networks. The
prototypical method is the Diffusion kernel (DK) \cite{ledk} inspired by the
heat diffusion phenomenon. The key idea is to allow a given amount of {\em
heat} placed on nodes to {\em diffuse} through the edges. The similarity
between two nodes $v_{i}, v_{j}$ is then defined as the amount of heat
starting from $v_{i}$ and reaching $v_{j}$ within a given time interval. In DK
the heat flow is proportional to the number of paths connecting two vertices,
which introduces an undesired bias that penalize peripheral nodes w.r.t.
central ones. This problem is tackled by a modified version of DK called
Markov exponential diffusion kernel (MED) \cite{medk} where a Markov matrix
replaces the Laplacian matrix.  Another kernel called Markov diffusion kernel
(MD) \cite{mdk}, exploits instead the notion of {\em diffusion distances}, a
measure of similarity between patterns of heat diffusion. The Regularized
Laplacian kernel (RL) \cite{rlk} represents instead a normalized version of
the random walk with restart model and defines the node similarity as the
number of paths connecting two nodes with different lengths. All these
approaches can be applied to dense networks with high degree nodes. A drawback
of these approaches is however their relatively low discriminative capacity.
This is in part due to the fact that information is processed in an additive
and independent fashion which prevents them from accurately modeling the
configuration of each gene's context. To address this issue here we propose to
employ a {\em decompositional} graph kernel (DGK) \cite{covolution_kernel}
technique. To exploit its higher discriminative capacity we first decompose
the network in a collection of connected sparse graphs and then we develop a
suitable kernel, that we call Conjunctive Disjunctive Node Kernel (CDNK).


\section{Method}\label{method} 

We start from the type of similarity notion computed by decomposition kernels
between graph instances and adapt it to express the similarity between nodes
in a single network. In this work we use three key ideas: 1) genes are labeled
using their functional profile, 2) we transform the network to distinguish
highly connected components from sparsely connected ones, and 3) we transform
the neighborhood of each gene in a sparse high dimensional vector that can be
easily processed by standard machine learning techniques such as SVMs.

\subsection{Gene Labeling} Gene-disease associations networks typically
represent genes as nodes labeled with a gene identifier. Here we take a
different approach: since we want to build a predictive system based on the
configuration of each gene's context, we use as labels a discretized
functional annotation based on the \textit{gene ontology} \cite{ontology}.
More precisely,  we use the gene ontology to construct binary vectors
representing the bag-of-words encoding for each gene (i.e. if a GO-term is
associated with the gene). The resulting representations are then clustered so
that genes with similar description profiles receive the same class identifier
as label.



\subsection{Network Decomposition} 

In gene-disease associations networks it is not uncommon to find nodes with
high degrees. Unfortunately these cases cannot be effectively processed by
decomposition kernels based on exact neighborhood matches\footnote{It is
improbable to find multiple cases of nodes with exactly the same, say 73
neighbors, and the similarity is defined by the number of shared identical
neighborhoods in our decomposition kernels.}. Instead we propose to decompose
the network in a linked collection of sparse sub-networks where each node has
a reduced connectivity. More precisely we distinguish two types of edges: {\em
conjunctive} and {\em disjunctive} edges. Nodes linked by conjunctive edges
are going to be used jointly to define the notion of context. Nodes linked by
disjunctive edges are instead used to define features based only on the
pairwise co-occurrence of the genes at the endpoints. The aim of the following
decompositions is to link sparse sub-networks (which comprise only conjunctive
edges) via disjunctive edges. 

\textit{Definitions.} A graph $G = (V,E)$ is a structure that
consists of a node set $V(G)$ and an edge set $E(G)$. The
\textit{distance} between two vertices $u$ and $v$, notated as
$\mathcal{D}(u,v)$, is the length of the shortest path between them. The
\textit{neighborhood} with radius $r$ of a vertex $v$ is the set of vertices
at a distance no greater than $r$ from $v$ and denoted by $N_r(v)$. 
The \textit{neighborhood subgraph} with radius $r$ of vertex $v$, denoted by
$\mathcal{N}_{r}^{v}$,  is the subgraph formed by the nodes in the
neighborhood with radius $r$ of $v$ and the relative edges with endpoints in
$N_r(v)$.

\textit{K-core decomposition \cite{kcore}}: The node set is partitioned in two
groups on the basis of the degree of each vertex w.r.t. a threshold degree
$D$. The vertex partition is used to induce the conjunctive vs disjunctive
edge partition: edges that have endpoints in the same part are marked as
conjunctive, while edges with endpoints in different parts are marked as
disjunctive. We apply the K-core decomposition iteratively considering only
the graph induced by the conjunctive edges until no node has a degree greater
than $D$\footnote{The degree is defined only considering incident conjunctive
edges}.

\textit{Clique decomposition \cite{clique}}:   All the cliques (completely
connected subgraphs) with a number of nodes greater than a threshold size $C$
are identified. For each clique a new 'representative' node is added to the
network. The endpoints of all edges incident on the clique's nodes are
transferred to the representative node. Disjunctive edges are introduced to
connect each node in the clique to the representative. Finally all edges with
both endpoints in the clique are removed.

In our work a network is transformed by applying first the k-core decomposition
and then the clique decomposition.

\subsection{Node Graph Kernels} 

We start from the Neighborhood Subgraph Pairwise Distance Kernel (NSPDK)
\cite{nspdk} and adapt it to express the similarity between nodes in a single
network. The key idea in NSPDK is to decompose graphs in small fragments and
count how many pairs of fragments are shared between two instances. We
introduce two improvements: 1) we partition the features according to the
individual node's neighborhood, and 2) feature construction distinguishes
between disjunctive and conjunctive edges.


\subsubsection{The Neighborhood Subgraph Pairwise Distance Kernel}

The NSPDK is an instance of convolution kernel \cite{covolution_kernel} where
given a graph $G \in \mathcal{G}$ and two rooted graphs $A_u, B_v$, the
relation $R_{r,d}(A_u, B_v, G)$ is true iff $A_u \cong \mathcal{N}_r^u$ is (up to isomorphism $\cong$) a neighborhood subgraph of radius $r$ of $G$ and so is $B_v \cong  \mathcal{N}_r^v$, with roots at distance
$\mathcal{D}(u,v)= d$. We denote $R^{-1}$ as the inverse relation that returns
all pairs of neighborhoods of radius $r$ at distance $d$ in $G$,
$R^{-1}_{r,d}(G) = \lbrace A_u, B_v | R_{r,d}(A_u,B_v,G)\rbrace$. The kernel
$\kappa_{r,d}$ over $\mathcal{G} \times \mathcal{G}$, counts the number of
such fragments in common in two input graphs: 

\begin{center}
$\kappa_{r,d}(G,G^{'}) = 
\!\!\!\!\!\!\!\!\!\!\!\! 
\sum\limits_{\substack{A_u, B_v \ \in \ R_{r,d}^{-1}(G) \\ 
{A'}_{u'}, {B'}_{v'} \ \in \ R_{r,d}^{-1}(G')
}} \!\!\!\!\!\!\!\!\!\!\!\!  { { \textbf{1}_{A_{u} \cong A'_{u'}}} \cdot {
\textbf{1}_{B_{v} \cong B'_{v'}}} }$, 
\end{center} 


where $\textbf{1}_{A \cong
B}$ is the \textit{exact matching function} that returns 1 if $A$ is
isomorphic to $B$ and 0 otherwise.  Finally, the NSPDK is defined as $K(G,G')
= \sum\limits_{r}{\sum\limits_{d}{\kappa_{r,d}(G,G')}}$, where for efficiency
reasons, the values of $r$ and $d$ are upper bounded to a given $r^*$ and
$d^*$, respectively.

\subsubsection{The Conjunctive Disjunctive Node Kernel}

We extend NSPDK and define a kernel $K(G_u,G_{u'})$ between two copies of the
same network $G$ where we distinguish the nodes $u$ and $u'$ respectively. The
idea is to define the features of a node $u$ as the subset of NSPDK features
that always have the vertex $u$ as one of the roots. In addition we
distinguish between two types of edges, termed {\em conjunctive} and {\em
disjunctive} edges. We consider only conjunctive edges when computing
distances and hence when we induce neighborhood subgraphs. When choosing the
pair of neighborhoods to form a single feature, we additionally consider roots
$u$ and $v$ that are not at distance $d$ but such that $u$ is connected to $w$
via a disjunctive edge and such that $w$ is at distance $d$ from $v$. In this
way disjunctive edges can still allow an 'information flow' even if their
endpoints are only considered in a pairwise fashion and not jointly.

Formally, we define two relations: the \textit{conjunctive relation}
$R^{\wedge}_{r,d}(A_u, B_v, G_u)$\footnote{This is identical to the NSPDK relation $R_{r,d}(A_u, B_v, G)$ .}  is true iff (i) $A_u \cong \mathcal{N}_r^u$ is a neighborhood subgraph of radius $r$ of $G_u$ and so is $B_v \cong \mathcal{N}_r^u$,  and (ii) $\mathcal{D}(u,v)= d$; the
\textit{disjunctive relation} $R_{r,d}^{\vee}(A_u, B_v, G)$ is true iff (i)  $A_u \cong \mathcal{N}_r^u$, $B_v \cong \mathcal{N}_r^u$
is true, (ii) there
exists a vertex $w$ such that $\mathcal{D}(w,v)= d$, and (iii) $(u,w)$ is a
disjunctive edge. We define $\kappa_{r,d}$ on the  inverse relations ${R^{\wedge}_{r,d}}^{ -1}$
and ${R^{\vee}_{r,d}}^{ -1}$

\begin{center}
 $\kappa_{r,d}(G_u,G_{u'}) = \!\!\!\!\!\!\!\!\!\!\!\!
 \sum\limits_{\substack {A_u,\ {B}_{v} \ \in \ ({R_{r,d}^{\wedge}}^{ -1}(G_u) \\ A'_{u'},\ {B'}_{v'} \ \in \ ({R_{r,d}^{\wedge}}^{ -1}(G_{u'}) }} \!\!\!\!\!\!\!\!\!\!\!\!
  { \textbf{1}_{A_u \cong A_{u'}} \cdot { \textbf{1}_{B_{v} \cong B'_{v'}}}}
+ \!\!\!\!\!\!\!\!\!\!\!\!
 \sum\limits_{\substack {A_u,\ {B}_{v} \ \in \ ({R_{r,d}^{\vee}}^{ -1}(G_u) \\
  A'_{u'},\ {B'}_{v'} \ \in \ ({R_{r,d}^{\vee}}^{ -1}(G_{u'}) }} \!\!\!\!\!\!\!\!\!\!\!\!
  { \textbf{1}_{A_u \cong A_{u'}} \cdot { \textbf{1}_{B_{v} \cong B'_{v'}}}}
  $,
\end{center}
$\kappa_{r,d}$ counts the number of identical pairs of neighboring graphs of radius $r$ at a distance $d$ between two vertices. The CDNK is finally defined as 
$K(G_u,G_v) = \sum\limits_{r}{\sum\limits_{d}{\kappa_{r,d}(G_u,G_v)}}.$
where once again for efficiency reasons,
the values of $r$ and $d$ are upper bounded to a given $r^*$ and $d^*$.

\section{Evaluation}
\label{evaluation}
We perform an empirical evaluation of the predictive performance of several kernel based methods on two of the databases used in \cite{medk}.

\textbf{BioGPS:} A gene co-expression network is constructed from BioGPS dataset, which contains 79 tissues, measured with the Affymetrix U133A array. Edges are inserted when the pairwise Pearson correlation coefficient (PCC) between genes is larger than 0.5.

\textbf{Pathways:} Pathway information is retrieved from KEGG, Reactome, PharmGKB and the Pathway Interaction Database. If a couple of proteins co-participate in any pathway, the two corresponding genes are linked.  

To evaluate the performance of graph node kernels we analyze the {\em gene
prioritization problem}. Given a set of genes known to be associated to a
given disease, gene prioritization is the task to rank the candidate genes
based on their probabilities to be related to that disease. Similar to the
evaluation process used in \cite{medk}, we choose 12 diseases with at least 30
confirmed genes. For each disease, we construct a positive set $\mathcal{P}$
with all confirmed disease genes, and a negative set $\mathcal{N}$ which
contains random genes associated at least to one disease class which is not
related to the class that is defining the positive set. In \cite{medk} the
ratio between the dataset sizes is chosen as $\vert \mathcal{N} \vert =
\frac{1}{2} \vert \mathcal{P} \vert$. 
The predictive performance of each method is evaluated via a leave-one-out
cross validation: one gene is kept out in turn and the rest
are used to train an SVM model. 
We compute a decision score $q_i$ for the
test gene $g_i$ as the top percentage value of score $s_i$ among all scores.
We collect all decision scores for every gene to
form a global decision score list on which we compute the AUC ROC.

\textbf{Model Selection.}
The hyper parameters of the various methods are set using a k-fold on a
dataset set that is then never used in the predictive performance estimate. We
try the values for diffusion parameter in DK and MED in $\lbrace 10^{-3},
10^{-3}, 10^{-2}, 10^{-1} \rbrace$, for time steps in MD in $\lbrace 1, 10,
100 \rbrace$ and for RL parameter in $\lbrace 1, 4, 7 \rbrace$. For CDNK, we
try for the degree threshold values in $\lbrace 10,\ 15,\ 20 \rbrace$, clique size
threshold in $\lbrace 4,\ 5 \rbrace$, maximum radius in $\lbrace 1,\ 2
\rbrace$, maximum distance in $\lbrace 2,\ 3,\ 4 \rbrace$. Finally, the $C$ of
SVM is searched in $\lbrace 10^{-5},  \ 10^{-4}, \ 10^{-3},\ 10^{-2},\ 10^{-1},\
1,\ 10,\ 10^2 \rbrace$.

\section{Results and Discussion}
\label{results_discussion}
\begin{table}
\centering
\setlength{\tabcolsep}{1mm}
\begin{tabular}{|c|c|c|c|c|c|?c|c|c|c|c|}
\hline
         & \multicolumn{5}{c|?}{\textbf{BioGPS}} & \multicolumn{5}{c|}{\textbf{Pathways}}\\
 \hline
Disease & K1 & K2 & K3 & K4 & K5 & K1 & K2 & K3 & K4 & K5\\
%Disease & LE & MD & ME & RL & CD & LE & MD & ME & RL & CD\\
 \hline
 1   & 52/5 & 57/4 & 59/3 & 59/2 & \textbf{65/1} 
 & 75/5 & 76/4 & 79/3 & 79/2 & \textbf{80/1} \\[0.5ex]
 
 2	 & 82/2 & 79/3 & 75/4 & 75/5 & \textbf{88/1}
 & 55/5 & 65/4 & 77/3 & 77/2 & \textbf{81/1} \\[0.5ex]

 3	& 64/4 & 60/5 & 72/2 & 72/1 & 66/3
 & 55/5 & 63/4 & 64/3 & 66/2 & 67/1 \\[0.5ex]
 
 4	 & 65/4 & 58/5 & 68/3 & 68/2 & \textbf{72/1}
 & 54/5 & 65/4 & \textbf{74/1} & 74/2 & 66/3 \\[0.5ex]
 
 5	 & 64/5 & 64/4 & 67/2 & 66/3 & \textbf{76/1}
 & 53/5 & 56/4 & 63/2 & 63/3 & \textbf{68/1} \\[0.5ex]				

 6	& 75/2 & 70/5 & 71/4 & 71/3 & \textbf{79/1} 
 & 83/5 & 93/4 & 97/2 & \textbf{97/1} & 93/3 \\[0.5ex]

 7	 & 73/3 & 67/5 & 75/2 & \textbf{76/1} & 69/4
 & 85/5 & 88/4 & 89/2 & \textbf{90/1} & 89/3 \\[0.5ex]

 8	 & 74/5 & \textbf{77/1} & 76/3 & 76/2 & 75/4
 & 54/5 & 66/4 & 72/3 & 72/2 & \textbf{73/1} \\[0.5ex]
 
 9	 & \textbf{72/1} & 66/5 & 68/3 & 70/2 & 67/4
 & 53/5 & 65/2 & 64/4 & 64/3 & \textbf{81/1} \\[0.5ex]

 10	 & 54/3 & 50/5 & 56/2 & 51/4 & \textbf{78/1}
 & 69/3 & 65/5 & 74/2 & \textbf{74/1} & 67/4 \\ [0.5ex]
 
 11	 & 58/4 & 51/5 & 59/3 & 59/2 & \textbf{72/1}
 & 54/5 & 69/4 & 75/2 & 74/3 & \textbf{77/1} \\ [1ex]

 \hline 
$\overline{AUC}$ & 66.6 & 63.5 & 67.8 & 67.5 & \textbf{73.3 }
 & 62.7 & 70.1 & 75.3 & 75.5 & \textbf{76.5} \\ [0.5ex]

$\overline{Rank}$ & 3.5 & 4.3 & 2.8 & 2.5 & \textbf{2.0}
 & 4.8 & 3.9 & 2.5 & 2.0 & \textbf{1.8}\\
 \hline 
\end{tabular}
\caption{\textit {Predictive performance on 11 gene-disease associations using networks induced by the BioGPS and the Pathway database. We report the AUC ROC and the rank for each kernel method: K1 = DK, K2 = MD, K3 = MED, K4 = RL, K5 = CDNK.}}
\label{table:results}
\end{table}

Table \ref{table:results} shows the AUC performance of the models trained by
using different graph node kernels on 11 gene-disease association problems
using the BioGPS and Pathways datasets to materialize the gene relation
network. In the table, the best result on each disease is marked in bold. We
note that CDNK ranks first in 7 out of 11 cases using  both networks. CDNK is
the top performant kernel also when considering the average AUC ROC and the
average rank with 73.3/2.0, 76.5/1.8, with a difference w.r.t. the second best
of 5.5$\%$ and 1$\%$ on BioGPS and Pathways, respectively. MED and RL show
similar and moderate results with small gap between them. DK and MD exhibit
modest performance on average and  are ranked last in many cases: 7
times out of 11 for MD in BioGPS and 10 out of 11 for DK in Pathways. 
Finally note that although CDNK has state of the art performance, there are
cases that would lead to a significant decrease in the quality of the results,
namely when networks have high {\em average} node degree as this would lead to
very sparse and fragmented decompositions.

\section{Conclusions} \label{conclusions} We have shown how decomposing a
network in a set of connected sparse graphs allows us to take advantage of the
discriminative power of CDNK, a novel decomposition kernel, to achieve state-
of-the-art results. In future work we will investigate how to extend the CDNK
approach to network problems when  multiple heterogeneous information sources
are available. 

\begin{footnotesize}

% IF YOU DO NOT USE BIBTEX, USE THE FOLLOWING SAMPLE SCHEME FOR THE REFERENCES
% ----------------------------------------------------------------------------
\begin{thebibliography}{99}

\bibitem{covolution_kernel} Haussler, David. Convolution kernels on discrete structures. Vol. 646. Technical report, Department of Computer Science, University of California at Santa Cruz, 1999.

\bibitem{ledk} Kondor, Risi Imre, and John Lafferty, Diffusion kernels on graphs and other discrete input spaces. \emph{ICML}. Vol. 2. 2002.

\bibitem{medk} Chen. B, et al. Disease gene identification by using graph kernels and Markov random fields. Science China Life Sciences 57.11 (2014): 1054-1063.

\bibitem{mdk} Fouss F, et al. An experimental investigation of graph kernels on a collaborative recommendation task. Proceedings of the 6th international conference on data mining 2006. ICDM 2006, 863-868.

\bibitem{rlk} Chebotarev P and Shamis E. The matrix forest therem and measuring relations in small social groups. Automation and Remote Control 1997, 58(9):1505-1514.

\bibitem{nspdk} C. Fabrizio, and K. De Grave, Fast neighborhood subgraph pairwise distance kernel. Proceedings of the 26th, International Conference on Machine Learning. Omnipress, 2010.

\bibitem{diseases} Goh KI et al, The human disease network. Proceedings of the National Academy of Sciences 104.21 (2007): 8685-8690.

\bibitem{kcore} A. Hamelin, et al, K-core decomposition: A tool for the visualization of large scale networks. arXiv preprint cs/0504107 (2005).

\bibitem{clique} R. E. Tarjan, Decomposition by clique separators, Discrete Math, vol. 55, no. 2, pp. 221-232, July. 1985.

\bibitem{ontology} Gene Ontology Consortium. The Gene Ontology (GO) database and informatics resource.
Nucleic acids research,(2004) 32(suppl 1), D258-D261.

\end{thebibliography}
% ----------------------------------------------------------------------------

% IF YOU USE BIBTEX,
% - DELETE THE TEXT BETWEEN THE TWO ABOVE DASHED LINES
% - UNCOMMENT THE NEXT TWO LINES AND REPLACE 'Name_Of_Your_BibFile'

%\bibliographystyle{unsrt}
%\bibliography{Name_Of_Your_BibFile}

\end{footnotesize}

% ****************************************************************************
% END OF BIBLIOGRAPHY AREA
% ****************************************************************************

\end{document}
