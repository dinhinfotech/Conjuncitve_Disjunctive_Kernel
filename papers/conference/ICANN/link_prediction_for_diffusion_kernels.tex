
%%%%%%%%%%%%%%%%%%%%%%% file typeinst.tex %%%%%%%%%%%%%%%%%%%%%%%%%
%
% This is the LaTeX source for the instructions to authors using
% the LaTeX document class 'llncs.cls' for contributions to
% the Lecture Notes in Computer Sciences series.
% http://www.springer.com/lncs       Springer Heidelberg 2006/05/04
%
% It may be used as a template for your own input - copy it
% to a new file with a new name and use it as the basis
% for your article.
%
% NB: the document class 'llncs' has its own and detailed documentation, see
% ftp://ftp.springer.de/data/pubftp/pub/tex/latex/llncs/latex2e/llncsdoc.pdf
%
%%%%%%%%%%%%%%%%%%%%%%%%%%%%%%%%%%%%%%%%%%%%%%%%%%%%%%%%%%%%%%%%%%%


\documentclass[runningheads,a4paper]{llncs}

\usepackage{amssymb}
\setcounter{tocdepth}{3}
\usepackage{graphicx}
\usepackage{color, colortbl}
\usepackage{float}
\usepackage{url}
\urldef{\mailsa}\path|{alfred.hofmann, ursula.barth, ingrid.haas, frank.holzwarth,|
\urldef{\mailsb}\path|anna.kramer, leonie.kunz, christine.reiss, nicole.sator,|
\urldef{\mailsc}\path|erika.siebert-cole, peter.strasser, lncs}@springer.com|    
\newcommand{\keywords}[1]{\par\addvspace\baselineskip
\noindent\keywordname\enspace\ignorespaces#1}

\begin{document}

\mainmatter  % start of an individual contribution

% first the title is needed
\title{Link Enrichment for Diffusion-based\\Graph Node Kernels}

% a short form should be given in case it is too long for the running head
\titlerunning{Link Enrichment for Diffusion-based Graph Node Kernels}

% the name(s) of the author(s) follow(s) next
%
% NB: Chinese authors should write their first names(s) in front of
% their surnames. This ensures that the names appear correctly in
% the running heads and the author index.
%
\author{Dinh Tran-Van$^1$ \and Alessandro Sperduti$^1$ \and Fabrizio Costa$^2$}
%\thanks{Please note that the LNCS Editorial assumes that all authors have used
%the western naming convention, with given names preceding surnames. This determines
%the structure of the names in the running heads and the author index.}%

%
\authorrunning{Link Enrichment for Diffusion-based Graph Node Kernels}
% (feature abused for this document to repeat the title also on left hand pages)

% the affiliations are given next; don't give your e-mail address
% unless you accept that it will be published
\institute{$^1$ Department of Mathematics, Padova University\\
%via Trieste, 63, 35121 Padova, Italy\\
$^2$ Department of Computer Science, University of Exeter\\
%Exeter EX4 4QF, UK\\
$\lbrace$dinh, sperduti$\rbrace$@math.unipd.it, f.costa@exeter.ac.uk }%\\
%\url{http://www.math.unipd.it}}

%
% NB: a more complex sample for affiliations and the mapping to the
% corresponding authors can be found in the file "llncs.dem"
% (search for the string "\mainmatter" where a contribution starts).
% "llncs.dem" accompanies the document class "llncs.cls".
%

\toctitle{Lecture Notes in Computer Science}
\tocauthor{Authors' Instructions}
\maketitle{}

% !TeX root = link_prediction_for_diffusion_kernels.tex

\begin{abstract}

The notion of node similarity is key in many network processing techniques and
it is especially important in diffusion graph kernels. However, when the graph
structure is affected by noise in the form of missing links, similarities are
distorted  proportionally to the sparsity of the graph and to the fraction of
missing links. Here, we introduce the notion of {\em link enrichment},
that is, performing link prediction in order to improve the performance of
diffusion-based kernels. We empirically show a robust and large effect for the
combination of a number of link prediction and a number of diffusion kernel
techniques on several gene-disease association problems.

\keywords{
Graph kernels, diffusion kernels, link prediction.
}
\end{abstract}


\section{Introduction}

A powerful approach to process large heterogeneous sources of data is to use
graph encodings \cite{proceeding1} \cite{jour1} and then use graph-based
learning systems. In these systems the notion of node similarity is key. A
common approach is to resort to graph node kernels such as diffusion-based
kernels \cite{proceeding2} where the graph node kernel measures the proximity
between any pair of nodes by taking into account the paths that connect them.
However, when the graph structure is affected by noise in the form of missing
links, node similarities are distorted  proportionally to the sparsity of the
graph and to the fraction of missing links. Two of the main reasons for this
are that 1) the lower the average node degree is, the smaller the number of
paths through which information can travel, and 2) missing links can end up
separating a network into multiple disconnected components. In this case,
since  information cannot travel across disconnected components, the
similarity of nodes belonging to different components is null. To address
these problems we propose to solve a link prediction task prior to the node
similarity computation and start studying the question: {\em how can we
improve node similarity using link prediction?} In this work we review both
the link prediction literature and the diffusion kernel literature, select a
subset of approaches in both categories that seem well suited, focus on a set
of node predicting problems in the bioinformatics domain and empirically
investigate the effectiveness of the combination of these approaches on the given
predictive tasks. The encouraging result that we find is that all the
strategies for link prediction we examined consistently enhance the
performance on downstream predictive tasks, often significantly improving
state of the art results.

% !TeX root = link_prediction_for_diffusion_kernels.tex

\section{Notation and Background}
\label{background}
Let us consider an undirected graph $G = (V, E)$ in which $V$ represents  a set of entities (vertices)  and $E$ characterizes the entity relationships (links). The adjacency matrix $A$ is a symmetric matrix used to describe the direct links between vertices $v_{i}$ and $v_{j}$ in the graph. Any entry $A_{ij}$ is equal to 1 when there exists a link connecting $v_{i}$ and $v_{j}$, and is 0 otherwise. The Laplacian matrix $L$ is defined as $L = D-A$, where $D$ is the diagonal matrix with non-null entries equal to the summation over the corresponding row of the adjacency matrix, i.e. $D_{ii}=\sum_j A_{ij}$. %The rest of the paper are described under this notation convention. 
\subsubsection{Graph Node Kernels.}
%As the desire of having a good node similarity measure for building graph-based leanring systems, many graph node kernels have been introduced and applied. 
A graph node kernel is a kernel which defines the similarity between nodes in a graph. Formally, a graph node kernel, $k(\cdot,\cdot)$, is defined as $k: V \times V \longrightarrow \mathbb{R}$ such that $k$ is symmetric positive semidefinite. Graph node kernels have been applied in various fields such as recommender systems, disease gene prioritization, and so on. Most graph node kernels belong to one of  two popular frameworks: diffusion-based graph node kernels and decomposition graph node kernels. 

Diffusion-based kernels can be considered as modifications of the laplacian diffusion kernel \cite{proceeding2}. These kernels measure the node proximity between any couple of nodes by taking into account the paths connecting them. They normally show state-of-the-art performance when dealing with dense graphs because of their ability to capture a global similarity measure. However, they perform poorly  when facing  sparse graphs with a low number of links and a high number of disconnected components. In the following, we briefly describe some of the most popular diffusion-based graph node kernels.
\begin{itemize}
\item \textit{Laplacian exponential diffusion kernel (LEDK) \cite{proceeding2}:} This kernel is based on heat diffusion phenomenon: imagine to initialize each vertex with a given amount of heat and let it flow through the edges until an arbitrary instant of time. The similarity between any vertex couple $v_{i}$, $v_{j}$ is the amount of heat starting from $v_{i}$ and reaching $v_{j}$ within a given time. The LEDK kernel matrix is computed by:
\begin{equation}
K_{LEDK} = e^{-\beta L}\; ,
\end{equation}
where $\beta$ is the diffusion parameter used to control the rate of diffusion, and $e^{X}=\sum_{k=0}^{\infty} \frac{1}{k!}X^k$ refers to the matrix exponential for matrix $X$. Choosing a consistent value for $\beta$ is very important: on the one side, if $\beta$ is too small, the local information cannot be diffused effectively and, on the other side, if it is too large, the local information will be lost. $K_{LEDK}$ is positive semi-definite as proved in \cite{proceeding2}.

\item \textit{Markov exponential diffusion kernel (MEDK) \cite{proceeding3}:} In LEDK, similarity values between high degree vertices is generally higher compared to that between low degree ones. This could be problematic since peripheral nodes have unbalanced similarities with respect to central nodes. To make the strength of individual vertices comparable, a modified version of LEDK is introduced:
\begin{equation}
K_{MEDK} = e^{-\beta M}\; ,
\end{equation}
where $M = (D-A-nI)/n$ and \textit{n}, \textit{I} are the total number of vertices in graph and identity matrix, respectively.

\item \textit{Markov diffusion kernel (MDK) \cite{jour3}:} MDK exploits the idea of diffusion distance, which is a measure of how similar the pattern of heat diffusion is between a pair of initialized nodes. In other words, it expresses how much nodes ``influence'' each other in a similar fashion. From the transition matrix \textit{P} ($P = D^{-1} A$), we define $Z(t) = \frac{1}{t}\sum_{\tau=1}^{t} P^{\tau}$. MDK kernel matrix is then computed as follows:
\begin{equation}
K_{MDK} = Z(t) Z^{\top}(t)\; .
\end{equation}

\item \textit{Regularized Laplacian kernel (RLK) \cite{proceeding4}:} It represents a normalized version of the random walk with restart model. The kernel matrix is defined as:
\begin{equation}
K_{RLK} = \sum_{n=0}^{\infty}\beta^{n}(-L)^n\; ,
\end{equation}
where $\beta$ is again the diffusion parameter. RLK counts the paths connecting two nodes on the graph induced by taking \textit{-L} as the adjacency matrix, regardless of the path length. Thus, a non-zero value is assigned to any couple of nodes as long as they are connected by any indirect path.
\end{itemize}
Decomposition graph node kernels take the idea from \cite{proceeding5} in which the similarity function between two graphs can be formed by decomposing each graph into subgraphs and by devising a valid local kernel between the subgraphs. This idea is then adjusted to measure graph node similarity by considering the neighborhood subgraph rooted at a vertex as its graph to compute. To form this kind of kernel, the graph matching problem, or equivalently the graph isomorphic problem, needs to be solved, which is not known to be solvable in polynomial time nor to belong to the NP-complete complexity class. An advantage of using decomposition kernels is the possibility to have non-zero similarity values for node couples located in distinct disconnected components of a graph. A recent and effective decomposition graph node kernel is the Conjunctive and Disjunctive Node Kernel (CDNK), proposed in \cite{proceeding6}. Considering a couple of nodes $u$ and $v$, the CDNK kernel defines the similarity between them by taking into account the common pairwise neighborhood subgraphs rooted at $u$ and $v$.
\subsubsection{Link Prediction.}
\label{link-prediction}
Link prediction is a task that intends to recover the missing links or predict links that are present on graph in the future states of graph evolution. In other words, link prediction allows to make the ranking over all non-observed links. This ranking is based on scores which associate with non-observed links showing their likelihood to be considered as links of the graph. Many link prediction methods have been proposed in the literature and they are applied in different domains: recommendation systems, bioinformatics, network security, etc. These methods can be classified into different categories as discussed in \cite{jour2}: \textit{similarity-based algorithms}, \textit{maximum likelihood methods}, and \textit{probabilistic models}. Similarity-based methods assign for each non-observed link a score and this score is then directly used as the proximity between starting and ending nodes of that link. In maximum likelihood methods, some organizing principles of the network structure are assumed. Then, the likelihood of any non-observed link can be calculated according to corresponding rules and parameters. Probabilistic models aim at abstracting the underlying structure from the observed network, and predicting the missing links by using a learned model. Given a target graph G, the probabilistic model will optimize a built target function to establish a model composed of a group of parameters which can best fit the observed data of the target network.

The similarity-based methods are the most popular in use since they are much simpler to deal with (and computationally less demanding) than maximum likelihood methods and probabilistic models. To define the similarity, we can use vertices' attributes. However, these attributes are normally hidden. Therefore, most similarity-based approaches are based on the structured similarity. Among similarity-based methods, global similarity-based methods normally outperform local and semi-local similarity-based algorithms since global methods are able to capture the glocal similarity between nodes of graph. It is important to notice that all graph node kernels belong to similarity group, and all graph node kernels described in previous section: LEDK, MEDK, MDK, RLK, CDNK, are global ones. Therefore, in this paper, we employ graph node kernels also for link prediction task.
% !TeX root = link_prediction_for_diffusion_kernels.tex

\section{Method}
\label{method}

Often the relational information that defines the network structure is
incomplete because certain relations are not known at a given moment in time
or have not been yet investigated. When this happens the resulting networks
tend to become sparse and composed of several disconnected components.
Diffusion-based kernels are not suited in these cases and show a degraded
predictive capacity. Our key idea is to introduce a {\em link enrichment
phase} that can address both issues and enhance the performance of diffusion-based systems.

Given a link prediction algorithm $M$, a diffusion-based graph node kernel $K$
and a sparse graph $G=(V, E)$ in which $|V| = n$ and $|E| = m$, with $m
\approx n$ the link enrichment method consists of two phases:

\begin{itemize}

\item enrichment: the link prediction algorithm $M$ is used to score all possible
$\frac{n(n-1)}{2}-m$ missing links. The top scoring $t$ links are added to $E$
to obtain $E'$ that defines the new graph $G'=(V,E')$.

\item kernel computation: the diffusion-based graph node kernel $K$ is applied
to graph $G^{'}$ to compute the kernel matrix $K'$ which captures the similarities
between any couple of nodes, possibly belonging to different components in $G$. 
\end{itemize}

The kernel matrix $K'$ can be used directly by a kernelized learning
algorithm, such as a support vector machine, to make predictive inferences.

% !TeX root = link_prediction_for_diffusion_kernels.tex

\section{Empirical evaluation} \label{evaluation} 

To empirically study the answer to the question: {\em how can we improve node
similarity using link prediction?} we would need to define a taxonomy of
prediction problems on networks that make use of the notion of node similarity
and analyze which link prediction strategies can be effectively coupled with
specific node similarity computation techniques given a class of problem. In
this paper we start such endeavor restricting the type of predictive problems
to that of node ranking in the sub-domain of gene-disease association studies.
More in detail, the task, known as {\em gene prioritization}, consists in
ranking candidate genes based on their probabilities to be related to a
disease on the basis of a given a set of genes known to be associated to the
disease of interest. 
We have assessed our approach on the four following datasets.

\textbf{BioGPS:} a gene co-expression network (7311 nodes and 911294 edges) is
constructed from the BioGPS dataset, which contains 79 tissues, measured with
the Affymetrix U133A array. Edges are inserted when the pairwise Pearson
correlation coefficient (PCC) between genes is larger than 0.5.

\textbf{HPRD:} a database of curated proteomic information pertaining to
human proteins. It is derived from \cite{jour5} with 9,465 vertices and 37,039
edges. We employ the HPRD version used in \cite{jour6} in which they remove
some vertices so to have 7311 nodes and 30503 edges remaining. HPRD, and
BioGPS, are used in \cite{proceeding3}.

\textbf{Phenotype similarity:} in order to capture the relatedness of genes
from a phenotypic point of view, we resort to OMIM \cite{jour4} data and the
phenotype similarity conceived by Van Driel et al. \cite{jour5}. They define a
similarity among OMIM phenotypes based on the relevance and the frequency of
the Medical Subject Headings (MeSH) vocabulary terms in the corresponding OMIM
text documents. We converted this information into a graph by linking those
genes whose associated phenotypes have a maximal phenotypic similarity greater
than a fixed cut-off value. The weight of the link is the maximal similarity
among the phenotypes relative to the two considered genes. We set the
similarity cut-off by following \cite{jour5} with a similarity score greater
than $0.3$. Finally, we obtain a network with 3393 nodes and 144739 edges.

\textbf{Biogridphys:} This dataset represents the physical interactions
among proteins. The idea is that mutations can affect physical interactions by
changing proteins shape and their effect can propagate through protein
networks. We introduce a link between two genes if their products interact. As
a result, the achieved network consists of 15389 nodes and 155333 edges.

\subsection{Evaluation Method}

To evaluate the performance of the diffusion kernels, we proceed in the same
way as \cite{proceeding3}: we choose $14$ diseases with have at least $30$
confirmed genes. For each disease, we construct a positive set
$\mathcal{P}$ with all confirmed disease genes. To build the negative set
$\mathcal{N}$ instead, we randomly sample a set of genes that are associated at least
to one disease class, but not related to the class which defines the positive
set such that $\vert \mathcal{N} \vert = \frac{1}{2} \vert \mathcal{P} \vert$.
We replicate this procedure 5 times\footnote{Note that the positive set is held constant, while the negative set varies.} . We assess the performance of kernels through a
paradigm similar to 3-fold CV: each ($\mathcal{P}$ + $\mathcal{U}$) set is
partitioned into three folds, where one fold is used to train the model (via a
linear SVM) and the two folds are used to test. For each test gene $g_i$,
model returns a score $s_i$ showing its likelihood to be associated to the
disease. Next a decision score $q_i$ is computed as the top percentage value
of $s_i$ among all candidate gene scores. We collect all decision scores for
every test genes to compute AUC-ROC. The final performance on the disease
class is obtained by taking average over $3\times$5 trials.

\textbf{Model Selection}: The hyper parameters of the various methods are
set using a 3-fold on a dataset set that is then never used in the predictive
performance estimation. We try the values for LEDK and MEDK in $\lbrace  0.01,
0.05, 0.1 \rbrace$, time steps in MDK in $\lbrace 3, 5, 10 \rbrace$ and RLK
parameter in $\lbrace 0.01, 0.1, 1 \rbrace$. For CDNK, we try for the degree
threshold value in $\lbrace 10,\ 15,\ 20 \rbrace$, clique size threshold in
$\lbrace 4,\ 5 \rbrace$, maximum radius in $\lbrace 1,\ 2 \rbrace$, maximum
distance in $\lbrace 2,\ 3,\ 4 \rbrace$. Number of added links are set in
$\lbrace 40\%,\ 50\%,\ 60\%,\ 70\% \rbrace$ over total number of existing
links. Finally, the $C$ of SVM is searched in $\lbrace 10^{-4},  \ 10^{-3}, \
10^{-2},\ 10^{-1}, 1,\ 10,\ 10^2, \ 10^3,\ 10^4 \rbrace$.

% !TeX root = link_prediction_for_diffusion_kernels.tex


\definecolor{cadetgrey}{rgb}{0.8721875,0.8721875,0.8721875}
\newcolumntype{g}{>{\columncolor{cadetgrey}}c}

{\setlength{\extrarowheight}{2pt}
\begin{table*}[!htbp]
\vspace*{-0.5cm}
\centering
\caption{\textit {Predictive performance on 14 gene-disease associations using four different networks induced by the BioGPS, Biogridphys, Hprd and Omim. We report the average AUC-ROC (\%) and standard deviations for all difussion-based kernels with (+) and without (-) link enrichment.}}
\label{table:results1}
\setlength{\tabcolsep}{0.6mm}
\begin{tabular}{|c|c|g|c|g|c|g|c|g|}
\hline
 & \multicolumn{2}{c|}{\textbf{BioGPS}} & \multicolumn{2}{c|}{\textbf{Biogridphys}} & \multicolumn{2}{c|}{\textbf{Hprd}} & \multicolumn{2}{c|}{\textbf{Omim}}\\
 \hline
Disease & - & + & - & + & - & + & - & + \\
\hline
1 & 60.3$\pm$1.5 & 63.4$\pm$1.0 & 73.1$\pm$4.1 & 77.1$\pm$2.9 & 75.5$\pm$0.2 & 77.5$\pm$0.9 & 85.3$\pm$1.1 & 86.9$\pm$1.5 \\
2 & 53.7$\pm$1.4 & 63.4$\pm$3.8 & 56.6$\pm$3.4 & 61.3$\pm$4.1 & 57.1$\pm$0.9 & 60.2$\pm$1.8 & 75.0$\pm$2.2 & 76.5$\pm$2.4 \\
3 & 50.2$\pm$0.4 & 58.6$\pm$3.0 & 58.9$\pm$5.9 & 67.5$\pm$7.7 & 61.8$\pm$3.6 & 70.7$\pm$3.8 & 77.3$\pm$1.8 & 83.1$\pm$0.9 \\
4 & 61.5$\pm$0.9 & 72.2$\pm$2.2 & 65.7$\pm$4.1 & 74.6$\pm$4.2 & 67.3$\pm$1.1 & 71.9$\pm$2.2 & 90.2$\pm$1.2 & 92.1$\pm$1.2 \\
5 & 55.1$\pm$0.4 & 61.7$\pm$0.9 & 54.2$\pm$4.8 & 60.7$\pm$4.0 & 57.7$\pm$1.6 & 67.0$\pm$1.8 & 76.4$\pm$0.8 & 81.9$\pm$1.5 \\
6 & 60.8$\pm$0.9 & 67.9$\pm$2.2 & 60.6$\pm$3.6 & 65.9$\pm$3.5 & 66.8$\pm$1.3 & 71.9$\pm$2.3 & 79.9$\pm$2.4 & 83.3$\pm$1.2 \\
7 & 68.1$\pm$1.4 & 73.4$\pm$0.7 & 57.7$\pm$3.2 & 63.7$\pm$4.0 & 68.9$\pm$2.1 & 72.5$\pm$1.2 & 81.0$\pm$1.2 & 84.1$\pm$1.0 \\
8 & 69.2$\pm$2.3 & 74.0$\pm$2.2 & 68.1$\pm$3.6 & 72.6$\pm$2.5 & 76.6$\pm$2.2 & 80.3$\pm$2.8 & 85.4$\pm$2.2 & 91.0$\pm$1.0 \\
9 & 62.0$\pm$1.6 & 64.5$\pm$1.4 & 68.7$\pm$4.6 & 71.7$\pm$4.3 & 68.4$\pm$2.5 & 75.0$\pm$3.2 & 78.5$\pm$0.2 & 80.6$\pm$0.6 \\
10 & 67.5$\pm$2.9 & 72.9$\pm$1.8 & 58.8$\pm$3.2 & 66.1$\pm$3.8 & 65.8$\pm$3.4 & 74.4$\pm$2.6 & 86.1$\pm$0.6 & 87.8$\pm$0.3 \\
11 & 58.7$\pm$1.8 & 62.3$\pm$1.5 & 58.2$\pm$1.2 & 61.6$\pm$1.7 & 60.1$\pm$1.1 & 64.2$\pm$1.5 & 82.0$\pm$1.4 & 83.6$\pm$0.9 \\
12 & 64.0$\pm$1.3 & 73.6$\pm$1.7 & 59.3$\pm$2.1 & 67.0$\pm$2.8 & 60.8$\pm$1.1 & 68.8$\pm$2.8 & 82.0$\pm$1.8 & 85.9$\pm$1.7 \\
13 & 56.5$\pm$0.9 & 63.3$\pm$2.4 & 55.8$\pm$1.1 & 65.1$\pm$4.2 & 66.4$\pm$1.3 & 71.8$\pm$1.7 & 83.1$\pm$2.8 & 87.5$\pm$2.5 \\
14 & 55.2$\pm$0.3 & 62.3$\pm$1.2 & 55.6$\pm$1.6 & 63.5$\pm$4.0 & 66.3$\pm$2.3 & 71.1$\pm$2.8 & 97.4$\pm$0.1 & 99.0$\pm$0.4 \\
\hline
$\overline{AUC}$ & 60.2$\pm$0.3 & 66.7$\pm$1.2 & 60.8$\pm$1.6 & 67.0$\pm$4.0 & 65.7$\pm$2.3 & 71.2$\pm$2.8 & 82.8$\pm$0.1 & 86.0$\pm$0.4 \\
\hline
\end{tabular}
\end{table*}

\section{Results and Discussion}
\label{results-discussion}

In Table \ref{table:results1} we report a synthesis of all the experiments.
Each row represent a different disease, in the columns we consider the
different sources of information used to build the underlying network (BioGPS,
Biogridphys, Hprd, Omim). Note that each resource yields a graph with
different characteristic sparsity and number of components. We compare the
average AUC-ROC scores in two cases: plain diffusion kernel (denoted by a "-"
symbol) and diffusion kernel on a modified network (denoted by a "+" symbol)
which includes a set of novel edges identified by a link prediction system.
Here we report the aggregated results (a detailed breakdown is available in
the \textit{Appendix}\footnote{https://github.com/dinhinfotech/ICANN/blob/master/appendix.pdf}) where we
have averaged not only across a random choice of negative genes, but also
among the type of diffusion kernel and the type of link prediction. The
noteworthy result is how consistent the result is: each link prediction method
improves each diffusion kernel algorithm, and on average using link prediction
yields a 15\% to 20\% relative error reduction for diffusion-based methods.
What varies is the amount of improvement, which depends on the coupling
between the four elements: the disease, the information source, the link
prediction method and the diffusion kernel algorithm. In specific we obtain
that the largest improvement is obtained for disease 3 (connective) where we
have a maximum improvement of 20\% ROC points, while the minimum improvement
is for disease 8 (immunological) with a minimal improvement of 0\% ROC points.
On average the largest improvement is of 13\% ROC points, while the smallest
improvement is on average of 1\% ROC point. These results are of interest
since these diffusion kernel approaches are currently state-of-the-art
approaches for gene-disease prioritization, and hence a technique that
can offer a consistent improvement has important practical consequences.

\section{Conclusion and Future Work}
\label{conclusion}

% what is the contribution that is not trivial

In this paper we have proposed the notion of {\em link enrichment} for
diffusion kernels, that is, the idea of carrying out the computation of
information diffusion on a network that contains edges identified by link
prediction approaches. We have discovered a surprisingly robust signal that
indicates that diffusion-based node kernels consistently benefit from the
coupling with link prediction techniques of the similarity-based type.

In future work we will carry out a more fine grained analysis, defining a
taxonomy of prediction problems on networks that make use of the notion of
node similarity and analyze which link prediction strategies can be
effectively coupled with specific node similarity computation techniques for
a given problem class. In addition we will study the quantitative
relation between the degree of missingness and the size of the improvement
offered by prepending the link prediction to the node similarity assessment.
Finally, we will extend the analysis to the more complex case of kernel
integration and data fusion, i.e. when multiple heterogeneous information
sources are used jointly to define the predictive task.


% EITHER use the included BST file
% \bibliographystyle{splncs03}
% \bibliography{yourbibfile}

% OR include the cited references explicitly
\begin{thebibliography}{4}

\bibitem{proceeding1} Huang, Z., et al.: A graph-based recommender system for digital library. Proceedings of the 2nd ACM/IEEE-CS joint conference on Digital libraries. ACM, 2002. 

\bibitem{proceeding2} Kondor, R. I., and Lafferty J.: Diffusion kernels on graphs and o ther discrete structures." Machine Learning, Proceedings of the 19th International Conference (ICML 2002). 2002.

\bibitem{proceeding3} Chen, B., et al.: Disease gene identification by using graph kernels and Markov random fields. Science China. Life Sciences 57.11 (2014): 1054.

\bibitem{proceeding4} Chebotarev P. and Shamis E.: The matrix-forest theorem and measuring relations in small social groups. Automation and Remote Control 1997, 58(9):15051514.

\bibitem{proceeding5} Haussler, D.: Convolution kernels on discrete structures. Technical Report UCS-CRL-99-10, UC Santa Cruz, 1999.

\bibitem{proceeding6} Tran-Van, D., Sperduti, A., and Costa, F.: Conjunctive disjunctive node kernel. Proceedings of 25th European Symposium on Artificial Neural Networks, Computational Intelligence and Machine Learning, 2017.

\bibitem{proceeding7} Costa, F., and Kurt D.: Fast neighborhood subgraph pairwise distance kernel. Proceedings of the 26th International Conference on Machine Learning. Omnipress, 2010.

\bibitem{jour1} Ramadan, E., Sadiq A., and Rafiul H.: "Network topology measures for identifying disease-gene association in breast cancer." BMC bioinformatics 17.7 (2016): 274.

\bibitem{jour2} Lu, L., and Tao Z.: Link prediction in complex networks: A survey. Physica A: Statistical Mechanics and its Applications 390.6 (2011): 1150-1170.

\bibitem{jour3} Fouss, F., et al.: An experimental investigation of kernels on graphs for collaborative recommendation and semisupervised classification. Neural Networks 31 (2012): 53-72.

\bibitem{jour4} McKusick, Victor A.: Mendelian Inheritance in Man and its online version, OMIM. The American Journal of Human Genetics 80.4 (2007): 588-604.

\bibitem{jour5} Chatr-Aryamontri, Andrew, et al.: The BioGRID interaction database: 2015 update. Nucleic acids research 43.D1 (2015): D470-D478.

\bibitem{jour6} Prasad, TSK et al.: Human Protein Reference Database - 2009 Update. Nucleic Acids Res 2009, 37(Database):D767-72.

\bibitem{jour5} Van Driel, Marc A., et al.: A text-mining analysis of the human phenome." European journal of human genetics 14.5 (2006): 535-542.

\end{thebibliography}

\end{document}
