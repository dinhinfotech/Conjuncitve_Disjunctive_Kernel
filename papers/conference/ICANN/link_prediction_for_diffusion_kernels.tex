
%%%%%%%%%%%%%%%%%%%%%%% file typeinst.tex %%%%%%%%%%%%%%%%%%%%%%%%%
%
% This is the LaTeX source for the instructions to authors using
% the LaTeX document class 'llncs.cls' for contributions to
% the Lecture Notes in Computer Sciences series.
% http://www.springer.com/lncs       Springer Heidelberg 2006/05/04
%
% It may be used as a template for your own input - copy it
% to a new file with a new name and use it as the basis
% for your article.
%
% NB: the document class 'llncs' has its own and detailed documentation, see
% ftp://ftp.springer.de/data/pubftp/pub/tex/latex/llncs/latex2e/llncsdoc.pdf
%
%%%%%%%%%%%%%%%%%%%%%%%%%%%%%%%%%%%%%%%%%%%%%%%%%%%%%%%%%%%%%%%%%%%


\documentclass[runningheads,a4paper]{llncs}

\usepackage{amssymb}
\setcounter{tocdepth}{3}
\usepackage{graphicx}
\usepackage{color, colortbl}
\usepackage{float}
\usepackage{url}
\urldef{\mailsa}\path|{alfred.hofmann, ursula.barth, ingrid.haas, frank.holzwarth,|
\urldef{\mailsb}\path|anna.kramer, leonie.kunz, christine.reiss, nicole.sator,|
\urldef{\mailsc}\path|erika.siebert-cole, peter.strasser, lncs}@springer.com|    
\newcommand{\keywords}[1]{\par\addvspace\baselineskip
\noindent\keywordname\enspace\ignorespaces#1}

\begin{document}

\mainmatter  % start of an individual contribution

% first the title is needed
\title{Link Enrichment for Diffusion-based\\Graph Node Kernels}

% a short form should be given in case it is too long for the running head
\titlerunning{Link Enrichment for Diffusion-based Graph Node Kernels}

% the name(s) of the author(s) follow(s) next
%
% NB: Chinese authors should write their first names(s) in front of
% their surnames. This ensures that the names appear correctly in
% the running heads and the author index.
%
\author{Dinh Tran-Van$^1$ \and Alessandro Sperduti$^1$ \and Fabrizio Costa$^2$}
%\thanks{Please note that the LNCS Editorial assumes that all authors have used
%the western naming convention, with given names preceding surnames. This determines
%the structure of the names in the running heads and the author index.}%

%
\authorrunning{Link Enrichment for Diffusion-based Graph Node Kernels}
% (feature abused for this document to repeat the title also on left hand pages)

% the affiliations are given next; don't give your e-mail address
% unless you accept that it will be published
\institute{$^1$ Department of Mathematics, Padova University\\
%via Trieste, 63, 35121 Padova, Italy\\
$^2$ Department of Computer Science, University of Exeter\\
%Exeter EX4 4QF, UK\\
$\lbrace$dinh, sperduti$\rbrace$@math.unipd.it, f.costa@exeter.ac.uk }%\\
%\url{http://www.math.unipd.it}}

%
% NB: a more complex sample for affiliations and the mapping to the
% corresponding authors can be found in the file "llncs.dem"
% (search for the string "\mainmatter" where a contribution starts).
% "llncs.dem" accompanies the document class "llncs.cls".
%

\toctitle{Lecture Notes in Computer Science}
\tocauthor{Authors' Instructions}
\maketitle{}

% !TeX root = link_prediction_for_diffusion_kernels.tex

\begin{abstract}

The notion of node similarity is key in many network processing techniques and
it is especially important in diffusion graph kernels. However, when the graph
structure is affected by noise in the form of missing links, similarities are
distorted  proportionally to the sparsity of the graph and to the fraction of
missing links. Here, we start to study the effectiveness of performing link
prediction  in order to improve the performance of diffusion-based kernels. We
empirically show a robust and large effect for the combination of a number of
link prediction and a number of diffusion kernel techniques on several
gene-disease association problems.

\keywords{
Graph kernels, diffusion kernels, link prediction.
}
\end{abstract}


\section{Introduction}

A powerful approach to process large heterogeneous sources of data is to use
graph encodings \cite{proceeding1} \cite{jour1} and then use graph-based
learning systems. In these systems the notion of node similarity is key. A
common approach is to resort to graph node kernels such as diffusion-based
kernels \cite{proceeding2} where the graph node kernel measures the proximity
between any pair of nodes by taking into account the paths that connect them.
However, when the graph structure is affected by noise in the form of missing
links, node similarities are distorted  proportionally to the sparsity of the
graph and to the fraction of missing links. Two of the main reasons for this
are that 1) the lower the average node degree is, the smaller the number of
paths through which information can travel, and 2) missing links can end up
separating a network into multiple disconnected components. In this case,
since  information cannot travel across disconnected components, the
similarity of nodes belonging to different components is null. To address
these problems we propose to solve a link prediction task prior to the node
similarity computation and start studying the question: {\em how can we
improve node similarity using link prediction?} In this work we review both
the link prediction literature and the diffusion kernel literature, select a
subset of approaches in both categories that seem well suited, focus on a set
of node predicting problems in the bioinformatics domain and empirically
investigate the effectiveness of the combination of these approaches on the given
predictive tasks. The encouraging result that we find is that all the
strategies for link prediction we examined consistently enhance the
performance on downstream predictive tasks, often significantly improving
state of the art results.

% !TeX root = link_prediction_for_diffusion_kernels.tex

\section{Notation and Background}
\label{background}
Let us consider an undirected graph $G = (V, E)$ in which $V$ represents  a set of entities (vertices)  and $E$ characterizes the entity relationships (links). The adjacency matrix $A$ is a symmetric matrix used to describe the direct links between vertices $v_{i}$ and $v_{j}$ in the graph. Any entry $A_{ij}$ is equal to 1 when there exists a link connecting $v_{i}$ and $v_{j}$, and is 0 otherwise. The Laplacian matrix $L$ is defined as $L = D-A$, where $D$ is the diagonal matrix with non-null entries equal to the summation over the corresponding row of the adjacency matrix, i.e. $D_{ii}=\sum_j A_{ij}$. %The rest of the paper are described under this notation convention. 
\subsubsection{Graph Node Kernels.}
%As the desire of having a good node similarity measure for building graph-based leanring systems, many graph node kernels have been introduced and applied. 
A graph node kernel is a kernel which defines the similarity between nodes in a graph. Formally, a graph node kernel, $k(\cdot,\cdot)$, is defined as $k: V \times V \longrightarrow \mathbb{R}$ such that $k$ is symmetric positive semidefinite. Most graph node kernels belong to one of  two popular frameworks: diffusion-based graph node kernels and decomposition graph node kernels. 

Diffusion-based kernels can be considered as modifications of the laplacian diffusion kernel \cite{proceeding2}. These kernels measure the node proximity between any couple of nodes by taking into account the paths connecting them. They normally show state-of-the-art performance when dealing with dense graphs because of their ability to capture a global similarity measure. However, they perform poorly  when facing  sparse graphs with a low number of links and a high number of disconnected components. In the following, we briefly describe some of the most popular diffusion-based graph node kernels.
\begin{itemize}
\item \textit{Laplacian exponential diffusion kernel:} One of the most well-known kernels for graphs is the Laplacian exponential diffusion kernel (LEDK), as it is widely used for exploiting discrete structures in general and graphs in particular. On the basis of the heat diffusion dynamics, Kondor and Lafferty proposed LEDK in \cite{proceeding2}: imagine to initialize each vertex with a given amount of heat and let it flow through the edges until an arbitrary instant of time. The similarity between any vertex couple $v_{i}$, $v_{j}$ is the amount of heat starting from $v_{i}$ and reaching $v_{j}$ within the given time. Therefore, LEDK can capture the long range relationship between vertices of a graph to define the global similarities. The formula to compute the LEDK kernel matrix is:
\begin{equation}
K_{LEDK} = e^{-\beta L}\; ,
\end{equation}
where $\beta$ is the diffusion parameter used to control the rate of diffusion, and $e^{X}=\sum_{k=0}^{\infty} \frac{1}{k!}X^k$ refers to the matrix exponential for matrix $X$. Choosing a consistent value for $\beta$ is very important: on the one side, if $\beta$ is too small, the local information cannot be diffused effectively and, on the other side, if it is too large, the local information will be lost. $K_{LEDK}$ is positive semi-definite as proved in \cite{proceeding2}.

\item \textit{Markov exponential diffusion kernel:} In LEDK, similarity values between high degree vertices is generally higher compared to that between low degree ones. Intuitively, the more paths connect two vertices, the more heat can flow between them. This could be problematic since peripheral nodes have unbalanced similarities with respect to central nodes. In order to make the strength of individual vertices comparable, a modified version of LEDK is introduced by Chen et al in \cite{proceeding3}.  This kernel is dubbed Markov exponential diffusion kernel (MEDK) and its kernel matrix is obtained by the following formula:
\begin{equation}
K_{MEDK} = e^{-\beta M}\; .
\end{equation}
The difference with respect to the LEDK is the replacement of \textit{L} by the matrix $M = (D-A-nI)/n$ where \textit{n} is the total number of vertices in graph and I is the identity matrix. The role of $\beta$ is the same as for LEDK.

\item \textit{Markov diffusion kernel:} The original Markov diffusion kernel MDK is introduced by Fouss et al. \cite{jour3}. It exploits the idea of diffusion distance, which is a measure of how similar the pattern of heat diffusion is between a pair of initialized nodes. In other words, it expresses how much nodes ``influence'' each other in a similar fashion. If their diffusion ways are alike, the similarity will be high and, vice versa, it will be low if they diffuse differently. The corresponding kernel matrix is computed starting from the transition matrix \textit{P} ($P = D^{-1} A$) and by defining $Z(t) = \frac{1}{t}\sum_{\tau=1}^{t} P^{\tau}$, as follows:
\begin{equation}
K_{MDK} = Z(t) Z^{\top}(t)\; .
\end{equation}

\item \textit{Regularized Laplacian kernel:} Another popular graph node kernel function used in graph mining is the regularized Laplacian kernel (RLK). This kernel function was introduced by Chebotarev and Shamis in \cite{proceeding4} and represents a normalized version of the random walk with restart model. Its kernel matrix is defined as follows:
\begin{equation}
K_{RLK} = \sum_{n=0}^{\infty}\beta^{n}(-L)^n\; ,
\end{equation}
where the parameter $\beta$ is again the diffusion parameter. RLK counts the paths connecting two nodes on the graph induced by taking \textit{-L} as the adjacency matrix, regardless of the path length. Thus, a non-zero value is assigned to any couple of nodes as long as they are connected by any indirect path. $K_{RLK}$ remains a relatedness measure even when the diffusion factor is large, by virtue of the negative weights assigned to self-loops.
\end{itemize}
Decomposition graph node kernels take the idea from \cite{proceeding5} in which the similarity function between two graphs can be formed by decomposing each graph into subgraphs and by devising a valid local kernel between the subgraphs. This idea is then adjusted to measure graph node similarity by considering the neighborhood subgraph rooted at a vertex as its graph to compute. In order to form this kind of kernel,  the graph matching problem, or equivalently the graph isomorphic problem, need to be solved, which is not known to be solvable in polynomial time nor to belong to the NP-complete complexity class. An advantage of using decomposition kernels is the possibility to have non-zero similarity values for node couples located in distinct disconnected components of a graph. A recent and effective decomposition graph node kernel is the Conjunctive and Disjunctive Node Kernel (CDNK), proposed in \cite{proceeding6}. CDNK is an extension of NSPDK \cite{proceeding7}, which is an instance of convolution kernel (decomposition kernel). Considering a couple of nodes $u$ and $v$, the CDNK kernel defines the similarity between them by taking into account the common pairwise neighborhood subgraphs rooted at $u$ and $v$.
\subsubsection{Link Enrichment.}
\label{link-enrichment}
Link enrichment is a task that intends to add the most likely non-observed links into a graph. This task can be performed by first using a link prediction method to make a ranking over all non-observed links based on their probabilities to be actual links, and then the top non-observed links are added into the graph. A considerable number of link prediction methods have been proposed in the literature. These methods can be classified into different categories as discussed in \cite{jour2}: \textit{similarity-based algorithms}, \textit{maximum likelihood methods}, and \textit{probabilistic models}. Similarity-based methods assign for each non-observed link a score and this score is then directly used as the proximity between starting and ending nodes of that link. In maximum likelihood methods, some organizing principles of the network structure are assumed. Then, the likelihood of any non-observed link can be calculated according to corresponding rules and parameters. Probabilistic models aim at abstracting the underlying structure from the observed network, and to predict the missing links by using a learned model. Given a target graph G, the probabilistic model will optimize a built target function to establish a model composed of a group of parameters which can best fit the observed data of the target network.

In this paper, we employ five graph node kernels described in the previous section: LEDK, MEDK, MDK, RLK and CDNK for link prediction since they belong to global similarity-based group. There are two reasons for the use of global similarity-based methods. Firstly, among similarity-based algorithms, global ones show, in general, better results   than local and semi-local similarity-based algorithms. Secondly, similarity-based algorithms are much simpler to deal with (and computationally less demanding) than maximum likelihood methods and probabilistic models.
% !TeX root = link_prediction_for_diffusion_kernels.tex

\section{Method}
\label{method}
Information encoded in data is usually incomplete. This usually leads to the sparsity issue when using graphs to represent  data. As a consequence,  graph-based systems which use diffusion-based kernels show limited performances. Therefore, in this section, we describe our proposed method, based on link enrichment, to strengthen diffusion-based graph node kernels, so that the performance of graph-based systems can be improved.

Given a sparse graph $G=(V, E)$ in which $|V| = n$ and $|E| = m$, the proposed method consists of two phases:
\begin{itemize}
\item Link enrichment: in the first phase, starting from the graph $G$, we utilize a link prediction method to compute scores for all $\frac{n(n-1)}{2}-m$ candidate links. These scores represent  their probabilities to be considered as a link in the graph. The candidate links are then sorted based on their corresponding scores. The top $t$ links in the sorted link list are added into $G$ to obtain a new enriched graph $G^{'}$.
\item Kernel computation: in the second phase, we apply the chosen diffusion-based graph node kernel to the achieved graph $G^{'}$ to compute the kernel matrix which encodes the similarities between any couple of nodes. This kernel matrix can then  be used into graph kernel-based learning systems to make inference.
\end{itemize}
% !TeX root = link_prediction_for_diffusion_kernels.tex

\section{Empirical evaluation} \label{evaluation} 

To empirically study the answer to the question: {\em how can we improve node
similarity using link prediction?} we would need to define a taxonomy of
prediction problems on networks that make use of the notion of node similarity
and analyze which link prediction strategies can be effectively coupled with
specific node similarity computation techniques for each given class of
problems. In addition we should also study the quantitative relation between
the degree of missingness and the size of the improvement offered by
prepending the link prediction to the node similarity assessment. In this
paper we start such endeavor restricting the type of predictive problems to
that of node ranking in the sub-domain of gene-disease association studies
with a fixed but unknown degree of missingness given by the current medical
knowledge. More in detail, the task, known as {\em gene prioritization},
consists in ranking candidate genes based on their probabilities to be related
to a disease on the basis of a given a set of genes experimentally known to be
associated to the disease of interest. We have studied the proposed approach on the
following 4 datasets:

\textbf{BioGPS:} a gene co-expression network (7311 nodes and 911294 edges) 
constructed from the BioGPS dataset, which contains 79 tissues, measured with
the Affymetrix U133A array. Edges are inserted when the pairwise Pearson
correlation coefficient (PCC) between genes is larger than 0.5.

\textbf{HPRD:} a database of curated proteomic information pertaining to
human proteins. It is derived from \cite{jour5} with 9,465 vertices and 37,039
edges. We employ the HPRD version used in \cite{jour6} that contains 7311 nodes and 30503 edges. %HPRD, and BioGPS, are used in \cite{proceeding3}.

\textbf{Phenotype similarity:} we use the OMIM \cite{jour4} dataset and the
phenotype similarity notion introduced by Van Driel et al. \cite{jour5} based on the relevance and the frequency of
the Medical Subject Headings (MeSH) vocabulary terms in OMIM documents. We built the graph linking those
genes whose associated phenotypes have a maximal phenotypic similarity greater
than a fixed cut-off value. Following \cite{jour5}, we set the
similarity cut-off to $0.3$. The resulting network has 3393 nodes and 144739 edges.

\textbf{Biogridphys:} this dataset encodes known physical interactions
among proteins. The idea is that mutations can affect physical interactions by
changing the shape of proteins and their effect can propagate through protein
networks. We introduce a link between two genes if their products interact. The resulting network has 15389 nodes and 155333 edges.

\subsection{Evaluation Method}

To evaluate the performance of the diffusion kernels, we follow
\cite{proceeding3}: we choose $14$ diseases with have at least $30$ confirmed
genes. For each disease, we construct a positive set $\mathcal{P}$ with all
confirmed disease genes. To build the negative set $\mathcal{N}$, we randomly
sample a set of genes that are associated at least to one disease class, but
not related to the class which defines the positive set such that $\vert
\mathcal{N} \vert = \frac{1}{2} \vert \mathcal{P} \vert$. We replicate this
procedure 5 times\footnote{Note that the positive set is held constant, while
the negative set varies.}. We assess the performance of kernels via a 3-fold
CV, where, after partitioning  the dataset $\mathcal{P}$ + $\mathcal{U}$ in 3
folds, we use one fold for training and the two remaining folds for testing.
For each test gene $g_i$, the model returns a score $s_i$ proportional to the
likelihood of being associated to the disease. Next a decision score $q_i$ is
computed as the top percentage value of $s_i$ among all candidate gene scores.
We collect all decision scores for every test genes to compute the area under
the curve for the receiver operating characteristic (AUC-ROC).  The final
performance on the disease class is obtained by taking average over $3\times$5
trials.


\textbf{Model Selection}: The hyper parameters of the various methods are
set using a 3-fold on a dataset set that is then never used in the predictive
performance estimation. We try the values for LEDK and MEDK in $\lbrace  0.01,
0.05, 0.1 \rbrace$, time steps in MDK in $\lbrace 3, 5, 10 \rbrace$ and RLK
parameter in $\lbrace 0.01, 0.1, 1 \rbrace$. For CDNK, we try for the degree
threshold value in $\lbrace 10,\ 15,\ 20 \rbrace$, clique size threshold in
$\lbrace 4,\ 5 \rbrace$, maximum radius in $\lbrace 1,\ 2 \rbrace$, maximum
distance in $\lbrace 2,\ 3,\ 4 \rbrace$. 
The number of links used for the enrichment are chosen in $\lbrace 40\%,\ 50\%,\ 60\%,\ 70\% \rbrace$of the number of existing links. 
Finally, the regularization tradeoff $C$ for the SVM is chosen in $\lbrace 10^{-4}, 10^{-3}, 0^{-2},\ 10^{-1}, 1,\ 10,\ 10^2, \ 10^3,\ 10^4 \rbrace$.

% !TeX root = link_prediction_for_diffusion_kernels.tex

\section{Results and Discussion}
\label{results-discussion}
\definecolor{cadetgrey}{rgb}{0.8721875,0.8721875,0.8721875}
\newcolumntype{g}{>{\columncolor{cadetgrey}}c}

{\setlength{\extrarowheight}{2pt}
\begin{table*}[!htbp]
\vspace*{-0.5cm}
\centering
\caption{\textit {Predictive performance on 14 gene-disease associations using four different networks induced by the BioGPS, Biogridphys, Hprd and Omim. We report the average AUC-ROC (\%) and standard deviations for all difussion-based kernels with (B) and without (A) using link enrichment.}}
\label{table:results1}
\setlength{\tabcolsep}{0.6mm}
\begin{tabular}{|c|c|g|c|g|c|g|c|g|}
\hline
 & \multicolumn{2}{c|}{\textbf{BioGPS}} & \multicolumn{2}{c|}{\textbf{Biogridphys}} & \multicolumn{2}{c|}{\textbf{Hprd}} & \multicolumn{2}{c|}{\textbf{Omim}}\\
 \hline
Disease & A & B & A & B & A & B & A & B \\
\hline
1 & 60.3$\pm$1.5 & 63.4$\pm$1.0 & 73.1$\pm$4.1 & 77.1$\pm$2.9 & 75.5$\pm$0.2 & 77.5$\pm$0.9 & 85.3$\pm$1.1 & 86.9$\pm$1.5 \\
2 & 53.7$\pm$1.4 & 63.4$\pm$3.8 & 56.6$\pm$3.4 & 61.3$\pm$4.1 & 57.1$\pm$0.9 & 60.2$\pm$1.8 & 75.0$\pm$2.2 & 76.5$\pm$2.4 \\
3 & 50.2$\pm$0.4 & 58.6$\pm$3.0 & 58.9$\pm$5.9 & 67.5$\pm$7.7 & 61.8$\pm$3.6 & 70.7$\pm$3.8 & 77.3$\pm$1.8 & 83.1$\pm$0.9 \\
4 & 61.5$\pm$0.9 & 72.2$\pm$2.2 & 65.7$\pm$4.1 & 74.6$\pm$4.2 & 67.3$\pm$1.1 & 71.9$\pm$2.2 & 90.2$\pm$1.2 & 92.1$\pm$1.2 \\
5 & 55.1$\pm$0.4 & 61.7$\pm$0.9 & 54.2$\pm$4.8 & 60.7$\pm$4.0 & 57.7$\pm$1.6 & 67.0$\pm$1.8 & 76.4$\pm$0.8 & 81.9$\pm$1.5 \\
6 & 60.8$\pm$0.9 & 67.9$\pm$2.2 & 60.6$\pm$3.6 & 65.9$\pm$3.5 & 66.8$\pm$1.3 & 71.9$\pm$2.3 & 79.9$\pm$2.4 & 83.3$\pm$1.2 \\
7 & 68.1$\pm$1.4 & 73.4$\pm$0.7 & 57.7$\pm$3.2 & 63.7$\pm$4.0 & 68.9$\pm$2.1 & 72.5$\pm$1.2 & 81.0$\pm$1.2 & 84.1$\pm$1.0 \\
8 & 69.2$\pm$2.3 & 74.0$\pm$2.2 & 68.1$\pm$3.6 & 72.6$\pm$2.5 & 76.6$\pm$2.2 & 80.3$\pm$2.8 & 85.4$\pm$2.2 & 91.0$\pm$1.0 \\
9 & 62.0$\pm$1.6 & 64.5$\pm$1.4 & 68.7$\pm$4.6 & 71.7$\pm$4.3 & 68.4$\pm$2.5 & 75.0$\pm$3.2 & 78.5$\pm$0.2 & 80.6$\pm$0.6 \\
10 & 67.5$\pm$2.9 & 72.9$\pm$1.8 & 58.8$\pm$3.2 & 66.1$\pm$3.8 & 65.8$\pm$3.4 & 74.4$\pm$2.6 & 86.1$\pm$0.6 & 87.8$\pm$0.3 \\
11 & 58.7$\pm$1.8 & 62.3$\pm$1.5 & 58.2$\pm$1.2 & 61.6$\pm$1.7 & 60.1$\pm$1.1 & 64.2$\pm$1.5 & 82.0$\pm$1.4 & 83.6$\pm$0.9 \\
12 & 64.0$\pm$1.3 & 73.6$\pm$1.7 & 59.3$\pm$2.1 & 67.0$\pm$2.8 & 60.8$\pm$1.1 & 68.8$\pm$2.8 & 82.0$\pm$1.8 & 85.9$\pm$1.7 \\
13 & 56.5$\pm$0.9 & 63.3$\pm$2.4 & 55.8$\pm$1.1 & 65.1$\pm$4.2 & 66.4$\pm$1.3 & 71.8$\pm$1.7 & 83.1$\pm$2.8 & 87.5$\pm$2.5 \\
14 & 55.2$\pm$0.3 & 62.3$\pm$1.2 & 55.6$\pm$1.6 & 63.5$\pm$4.0 & 66.3$\pm$2.3 & 71.1$\pm$2.8 & 97.4$\pm$0.1 & 99.0$\pm$0.4 \\
\hline
$\overline{AUC}$ & 60.2$\pm$0.3 & 66.7$\pm$1.2 & 60.8$\pm$1.6 & 67.0$\pm$4.0 & 65.7$\pm$2.3 & 71.2$\pm$2.8 & 82.8$\pm$0.1 & 86.0$\pm$0.4 \\
\hline
\end{tabular}
\end{table*}
\newcommand{\squeezeup}{\vspace{-4mm}}
\squeezeup
Table \ref{table:results1} shows the average predictive performances of diease gene prioritization system on 14 genetic diseases and four genetic networks. In each experiment, we test performance of system with the use kernels with and without using link enrichment. Overall, the performance of system with link enrichment are remarkably higher than not using link enrichment in every disease and dataset. In particular, the use of link enrichment helps to improve in average around 6$\%$ on BioGPS, Biogridphys and Hprd, and 3$\%$ in Omim. The detail of all experimental results can be found in the \textit{appendix} \footnote{https://github.com/}. Considering a specific kernel, the performance of the system get higher with the use of any link enrichment methods comparing to it without using link enrichment. 
It illustrates that the method is stable for the use of link enrichment methods and is considerable when constructing graph-based learning systems using diffusion-based kernels.

\section{Conclusion}
\label{conclusion}
In this paper, we have proposed a novel method to boost the power of diffusion-based graph node kernel by using link enrichment paradigm. The results achieved from empirical experiments illustrate that our proposed method is noticeable when using diffusion-based graph node kernels to build learning systems. For the coming work, we desire to apply this boosting method to improve the performance of systems using kernel integration.



% EITHER use the included BST file
% \bibliographystyle{splncs03}
% \bibliography{yourbibfile}

% OR include the cited references explicitly
\begin{thebibliography}{4}

\bibitem{proceeding1} Huang, Z., et al.: A graph-based recommender system for digital library. Proceedings of the 2nd ACM/IEEE-CS joint conference on Digital libraries. ACM, 2002. 

\bibitem{proceeding2} Kondor, R. I., and Lafferty J.: Diffusion kernels on graphs and o ther discrete structures." Machine Learning, Proceedings of the 19th International Conference (ICML 2002). 2002.

\bibitem{proceeding3} Chen, B., et al.: Disease gene identification by using graph kernels and Markov random fields. Science China. Life Sciences 57.11 (2014): 1054.

\bibitem{proceeding4} Chebotarev P. and Shamis E.: The matrix-forest theorem and measuring relations in small social groups. Automation and Remote Control 1997, 58(9):15051514.

\bibitem{proceeding5} Haussler, D.: Convolution kernels on discrete structures. Technical Report UCS-CRL-99-10, UC Santa Cruz, 1999.

\bibitem{proceeding6} Tran-Van, D., Sperduti, A., and Costa, F.: Conjunctive disjunctive node kernel. Proceedings of 25th European Symposium on Artificial Neural Networks, Computational Intelligence and Machine Learning, 2017.

\bibitem{proceeding7} Costa, F., and Kurt D.: Fast neighborhood subgraph pairwise distance kernel. Proceedings of the 26th International Conference on Machine Learning. Omnipress, 2010.

\bibitem{jour1} Ramadan, E., Sadiq A., and Rafiul H.: "Network topology measures for identifying disease-gene association in breast cancer." BMC bioinformatics 17.7 (2016): 274.

\bibitem{jour2} Lu, L., and Tao Z.: Link prediction in complex networks: A survey. Physica A: Statistical Mechanics and its Applications 390.6 (2011): 1150-1170.

\bibitem{jour3} Fouss, F., et al.: An experimental investigation of kernels on graphs for collaborative recommendation and semisupervised classification. Neural Networks 31 (2012): 53-72.

\bibitem{jour4} McKusick, Victor A.: Mendelian Inheritance in Man and its online version, OMIM. The American Journal of Human Genetics 80.4 (2007): 588-604.

\bibitem{jour5} Chatr-Aryamontri, Andrew, et al.: The BioGRID interaction database: 2015 update. Nucleic acids research 43.D1 (2015): D470-D478.

\bibitem{jour6} Prasad, TSK et al.: Human Protein Reference Database - 2009 Update. Nucleic Acids Res 2009, 37(Database):D767-72.

\bibitem{jour5} Van Driel, Marc A., et al.: A text-mining analysis of the human phenome." European journal of human genetics 14.5 (2006): 535-542.

\end{thebibliography}

\end{document}
