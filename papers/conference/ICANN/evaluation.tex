\section{Evaluation}
\label{evaluation}
We have assessed our approach on four different datasets. In the following, we first describe the datasets and then we present the obtained results.
\subsection{Datasets}
The proposed method aims to strengthen the power of diffusion-based kernels when dealing with sparse graphs. Therefore, we employ  genetic-related data which typically lead to sparse graphs. Hereafter, we briefly describe them.

\textbf{BioGPS:} a gene co-expression network (7311 nodes and 911294 edges) is constructed from the BioGPS dataset, which contains 79 tissues, measured with the Affymetrix U133A array. Edges are inserted when the pairwise Pearson correlation coefficient (PCC) between genes is larger than 0.5.

\textbf{HPRD:} a database of curated proteomic information pertaining to human proteins. It is derived from \cite{jour5} with 9,465 vertices and 37,039 edges. We employ the HPRD version used in \cite{jour6} in which they remove some vertices so to have 7311 nodes and 30503 edges remaining. HPRD, and  BioGPS, are used in \cite{proceeding3}.

\textbf{Phenotype similarity:} in order to capture the relatedness of genes from a phenotypic point of view, we resort to OMIM \cite{jour4} data and the phenotype similarity conceived by Van Driel et al. \cite{jour5}. They define a similarity among OMIM phenotypes based on the relevance and the frequency of the Medical Subject Headings (MeSH) vocabulary terms in the corresponding OMIM text documents. We converted this information into a graph by linking those genes whose associated phenotypes have a maximal phenotypic similarity greater than a fixed cut-off value. The weight of the link is the maximal similarity among the phenotypes relative to the two considered genes. We set the similarity cut-off by following \cite{jour5} with a similarity score greater than $0.3$. Finally, we obtain a network with 3393 nodes and 144739 edges.

\textbf{Biogridphys:} This dataset represents the physical interactions among proteins. The idea is that mutations can affect physical interactions by changing proteins shape and their effect can propagate through protein networks. We introduce a link between two genes if their products interact. As a result, the achieved network consists of 15389 nodes and 155333 edges.

\subsection{Evaluation Methods}
To evaluate the performance of the considered kernels, we use the {\em gene prioritization} task, i.e. given a set of genes known to be associated to a given disease, gene prioritization consists in ranking the candidate genes based on their probabilities to be related to that disease. Similar to the evaluation process used in \cite{proceeding3}, we choose $14$ diseases with at least $30$ confirmed involved genes. For each disease, we first construct a positive set $\mathcal{P}$ with all confirmed disease genes, and a negative set $\mathcal{N}$ contains random genes associated at least to one disease class, but not related to the class which defines the positive set such that $\vert \mathcal{N} \vert = \frac{1}{2} \vert \mathcal{P} \vert$. We then repeat this procedure five times. Each time, we keep the positive set and only change the negative set. As a result, we have five different training sets for each disease class. We assess the performance of kernels through a paradigm similar to 3-fold CV: each ($\mathcal{P}$ + $\mathcal{U}$) set is partitioned into three folds, where one fold is used to train the model (via a linear SVM) and the two folds are used to test. For each test gene $g_i$, model returns a score $s_i$ showing its likelihood to be associated to the disease. Next a dicision score $q_i$ is computed as the top percentage value of $s_i$ among all candidate gene scores. We collect all decision scores for every test genes to compute AUC-ROC. The final performance on the disease class is obtained by taking average over $3\times$5 trials.

\textbf{Model Selection}: The hyper parameters of the various methods are set using a 3-fold on a dataset set that is then never used in the predictive performance estimation. We try the values for LEDK and MEDK in $\lbrace  0.01, 0.05, 0.1 \rbrace$, time steps in MDK in $\lbrace 3, 5, 10 \rbrace$ and RLK parameter in $\lbrace 0.01, 0.1, 1 \rbrace$. For CDNK, we try for the degree threshold value in $\lbrace 10,\ 15,\ 20 \rbrace$, clique size threshold in $\lbrace 4,\ 5 \rbrace$, maximum radius in $\lbrace 1,\ 2 \rbrace$, maximum distance in $\lbrace 2,\ 3,\ 4 \rbrace$. Number of added links are set in $\lbrace 40\%,\ 50\%,\ 60\%,\ 70\% \rbrace$ over total number of existing links. Finally, the $C$ of SVM is searched in $\lbrace 10^{-4},  \ 10^{-3}, \ 10^{-2},\ 10^{-1}, 1,\ 10,\ 10^2, \ 10^3,\ 10^4 \rbrace$.
