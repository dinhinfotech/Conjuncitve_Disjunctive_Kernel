\begin{abstract}

When processing networks it is important to be able to com- pare nodes. Diffusion graph kernels are an effective and flexible technique to define node similarities. However, when the underlying graphical structure is affected by noise in the form of missing links, the similarity notion computed can be distorted in a way that is proportional to the sparsity of the graph and the fraction of missing links. Here, we propose to add a step of link prediction in order to improve diffusion-based kernels. We empirically show a robust and large effect on gene-disease


Node similarity  is one of the key points which determines the performance of graph-based learning systems. Diffusion-based graph node kernels are commonly used in many applications to capture  node similarity. However, they only return state-of-the-art results in the case of dense graphs. In this paper, we propose a method employing link enrichment that aims to strengthen diffusion-based kernels when working with sparse graphs. The empirical assessment shows that our method considerably improves the power of diffusion-based graph node kernels in the case of sparse graphs. 

\keywords{Graph node kernels, diffusion-based kernels, strenghtening diffusion-based kernels, link enrichment.}
\end{abstract}


\section{Introduction}

Recently, with the fast development of science and technology, we have witnessed the rapid growth of data in terms of both volume and variety. In order to efficiently extract knowledge from this huge amount of data, a number of learning systems have been introduced. Some of these systems are tailored for specific types of data. Graph is a widely used data representation and it is employed by many systems in different domains \cite{proceeding1}, \cite{jour1}. Learning systems that take graphs as their input are referred to as graph-based learning systems.

In graph-based systems, the measurement of the proximity between nodes of a graph is one of the key factors that determines the performance of the system. The most common paradigm used to capture similarity between nodes is to resort to graph node kernels. In fact, many  graph node kernels have been proposed and applied in several real-world applications and domains. Among them, diffusion-based kernels \cite{proceeding2} are the most commonly employed\footnote{A diffusion-based graph node kernel measures the proximity between any couple of nodes by taking into account paths connecting them.}, very often returning state-of-the-art results. However, these node kernels usually show good performance only when dealing with dense graphs, i.e., graphs with a high value of average node degree. Vice versa in the case of sparse graphs, i.e. graphs with a low value of average node degree, they usually lead to poor performance. This is due to: {\it i)} the number of links in the graph is very limited compared to the situation encountered in a complete graph, so the information cannot be spreaded properly through the whole graph; {\it ii)} the lack of links also causes the fragmentation of the graph into isolated components. It is important to stress that, in case of fragmentation, information cannot be diffused between isolated components. Therefore, the similarity between nodes located in different isolated components, as measured by diffusion-based graph node kernels, is equal to zero. As a consequence, the performances of these type of kernels hinders the possibility to build good graph-based learning systems. To overcome this problem, we come up with the idea of using link enrichment. Link enrichment is a task that aims at predicting the most probable candidate links to be considered as missing links of a graph. Many link prediction methods have been proposed. In \cite{jour2}, a quite exhaustive study of link prediction methods is presented in which methods proposed in literature are classified into different groups. The most widely used framework is the similarity-based one because of its effectiveness and ease of use. In this group of methods, to each pair of nodes is assigned a score which is directly used as the similarity between nodes.

To the best of our knowledge, there is no investigation that has been done to boost the performance of diffusion-based kernels by using link enrichment. Therefore, in this paper, we present a method that goes along this direction.
%intends to strengthen the power of diffusion-based graph node kernels by employing link enrichment paradigm. 
The experimental assessment  on different real-world datasets confirms the efficacy of our proposed method.

%This paper is organized as follows: we first introduce the notaion and background in the section \ref{background}. We then describe our proposed method in section \ref{method}. The evaluation and results are presented in section \ref{evaluation} and section \ref{results-discussion}, respectively. Finally, the conclusion is writen in section \ref{conclusion}.
