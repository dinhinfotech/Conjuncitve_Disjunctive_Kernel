\begin{abstract}

The notion of node similarity is key in many network processing techniques and
it is especially important in diffusion graph kernels. However, when the graph
structure is affected by noise in the form of missing links, similarities are
distorted  proportionally to the sparsity of the graph and to the fraction of
missing links. Here, we start to study the effectiveness of performing link
prediction  in order to improve the performance of diffusion-based kernels. We
empirically show a robust and large effect for the combination of a number of
link prediction and a number of diffusion kernel techniques on several
gene-disease association problems.

\keywords{
Graph kernels, diffusion kernels, link prediction.
}
\end{abstract}


\section{Introduction}

A powerful approach to process large heterogeneous sources of data is to use
graph encodings \cite{proceeding1} \cite{jour1} and then use graph-based
learning systems. In these systems the notion of node similarity is key. A
common approach is to resort to graph node kernels such as diffusion-based
kernels \cite{proceeding2} where the graph node kernel measures the proximity
between any pair of nodes by taking into account paths connecting them.
However, when the graph structure is affected by noise in the form of missing
links, similarities are distorted  proportionally to the sparsity of the graph
and to the fraction of missing links. Two of the main reasons for this are
that 1) the lower the average node degree is, the smaller the number of paths
through which the information can travel is, and 2) missing links can end up
splitting a network into multiple disconnected components. In this case, since
information cannot travel across disconnected components, the similarity
between nodes belonging to different components is zero. To address these
problems we propose to solve  a link prediction problem prior to the node
similarity computation and start studying the question: {\em how can we
improve node similarity using link prediction?} In this work we review both
the link prediction literature and the diffusion kernel literature, select a
subset of approaches in both categories that seem well suited for the task,
fix a set of related node predicting problems and empirically investigate the
effectiveness of their combination. As a result we find that many strategies
for link prediction consistently enhance the performance on downstream
predictive tasks, often significantly improving state of the art results.
