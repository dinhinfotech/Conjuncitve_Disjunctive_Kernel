% !TeX root = link_prediction_for_diffusion_kernels.tex


\definecolor{cadetgrey}{rgb}{0.8721875,0.8721875,0.8721875}
\newcolumntype{g}{>{\columncolor{cadetgrey}}c}

{\setlength{\extrarowheight}{2pt}
\begin{table*}[!htbp]
\vspace*{-0.5cm}
\centering
\caption{\textit {Predictive performance on 14 gene-disease associations using four different networks induced by the BioGPS, Biogridphys, Hprd and Omim. We report the average AUC-ROC (\%) and standard deviations for all difussion-based kernels with (+) and without (-) link enrichment.}}
\label{table:results1}
\setlength{\tabcolsep}{0.6mm}
\begin{tabular}{|c|c|g|c|g|c|g|c|g|}
\hline
 & \multicolumn{2}{c|}{\textbf{BioGPS}} & \multicolumn{2}{c|}{\textbf{Biogridphys}} & \multicolumn{2}{c|}{\textbf{Hprd}} & \multicolumn{2}{c|}{\textbf{Omim}}\\
 \hline
Disease & - & + & - & + & - & + & - & + \\
\hline
1 & 60.3$\pm$1.5 & 63.4$\pm$1.0 & 73.1$\pm$4.1 & 77.1$\pm$2.9 & 75.5$\pm$0.2 & 77.5$\pm$0.9 & 85.3$\pm$1.1 & 86.9$\pm$1.5 \\
2 & 53.7$\pm$1.4 & 63.4$\pm$3.8 & 56.6$\pm$3.4 & 61.3$\pm$4.1 & 57.1$\pm$0.9 & 60.2$\pm$1.8 & 75.0$\pm$2.2 & 76.5$\pm$2.4 \\
3 & 50.2$\pm$0.4 & 58.6$\pm$3.0 & 58.9$\pm$5.9 & 67.5$\pm$7.7 & 61.8$\pm$3.6 & 70.7$\pm$3.8 & 77.3$\pm$1.8 & 83.1$\pm$0.9 \\
4 & 61.5$\pm$0.9 & 72.2$\pm$2.2 & 65.7$\pm$4.1 & 74.6$\pm$4.2 & 67.3$\pm$1.1 & 71.9$\pm$2.2 & 90.2$\pm$1.2 & 92.1$\pm$1.2 \\
5 & 55.1$\pm$0.4 & 61.7$\pm$0.9 & 54.2$\pm$4.8 & 60.7$\pm$4.0 & 57.7$\pm$1.6 & 67.0$\pm$1.8 & 76.4$\pm$0.8 & 81.9$\pm$1.5 \\
6 & 60.8$\pm$0.9 & 67.9$\pm$2.2 & 60.6$\pm$3.6 & 65.9$\pm$3.5 & 66.8$\pm$1.3 & 71.9$\pm$2.3 & 79.9$\pm$2.4 & 83.3$\pm$1.2 \\
7 & 68.1$\pm$1.4 & 73.4$\pm$0.7 & 57.7$\pm$3.2 & 63.7$\pm$4.0 & 68.9$\pm$2.1 & 72.5$\pm$1.2 & 81.0$\pm$1.2 & 84.1$\pm$1.0 \\
8 & 69.2$\pm$2.3 & 74.0$\pm$2.2 & 68.1$\pm$3.6 & 72.6$\pm$2.5 & 76.6$\pm$2.2 & 80.3$\pm$2.8 & 85.4$\pm$2.2 & 91.0$\pm$1.0 \\
9 & 62.0$\pm$1.6 & 64.5$\pm$1.4 & 68.7$\pm$4.6 & 71.7$\pm$4.3 & 68.4$\pm$2.5 & 75.0$\pm$3.2 & 78.5$\pm$0.2 & 80.6$\pm$0.6 \\
10 & 67.5$\pm$2.9 & 72.9$\pm$1.8 & 58.8$\pm$3.2 & 66.1$\pm$3.8 & 65.8$\pm$3.4 & 74.4$\pm$2.6 & 86.1$\pm$0.6 & 87.8$\pm$0.3 \\
11 & 58.7$\pm$1.8 & 62.3$\pm$1.5 & 58.2$\pm$1.2 & 61.6$\pm$1.7 & 60.1$\pm$1.1 & 64.2$\pm$1.5 & 82.0$\pm$1.4 & 83.6$\pm$0.9 \\
12 & 64.0$\pm$1.3 & 73.6$\pm$1.7 & 59.3$\pm$2.1 & 67.0$\pm$2.8 & 60.8$\pm$1.1 & 68.8$\pm$2.8 & 82.0$\pm$1.8 & 85.9$\pm$1.7 \\
13 & 56.5$\pm$0.9 & 63.3$\pm$2.4 & 55.8$\pm$1.1 & 65.1$\pm$4.2 & 66.4$\pm$1.3 & 71.8$\pm$1.7 & 83.1$\pm$2.8 & 87.5$\pm$2.5 \\
14 & 55.2$\pm$0.3 & 62.3$\pm$1.2 & 55.6$\pm$1.6 & 63.5$\pm$4.0 & 66.3$\pm$2.3 & 71.1$\pm$2.8 & 97.4$\pm$0.1 & 99.0$\pm$0.4 \\
\hline
$\overline{AUC}$ & 60.2$\pm$0.3 & 66.7$\pm$1.2 & 60.8$\pm$1.6 & 67.0$\pm$4.0 & 65.7$\pm$2.3 & 71.2$\pm$2.8 & 82.8$\pm$0.1 & 86.0$\pm$0.4 \\
\hline
\end{tabular}
\end{table*}

\section{Results and Discussion}
\label{results-discussion}

In Table \ref{table:results1} we report a synthesis of all the experiments.
Each row represent a different disease, in the columns we consider the
different sources of information used to build the underlying network (BioGPS,
Biogridphys, Hprd, Omim). Note that each resource yields a graph with
different characteristic sparsity and number of components. We compare the
average AUC-ROC scores in two cases: plain diffusion kernel (denoted by a "-"
symbol) and diffusion kernel on a modified network (denoted by a "+" symbol)
which includes a set of novel edges identified by a link prediction system.
Here we report the aggregated results (a detailed breakdown is available in
the \textit{Appendix}\footnote{https://github.comXXXXXXXXXXXXXXXX/}) where we
have averaged not only across a random choice of negative genes, but also
among the type of diffusion kernel and the type of link prediction. The
noteworthy result is how consistent the result is: each link prediction method
improves each diffusion kernel algorithm, and on average using link prediction
yields a 15\% to 20\% relative error reduction for diffusion-based methods.
What varies is the amount of improvement, which depends on the coupling
between the four elements: the disease, the information source, the link
prediction method and the diffusion kernel algorithm. In specific we obtain
that the biggest improvement is obtained for disease 3 () we have a maximum
improvement of 20\% ROC points, while for disease 8 () the minimal improvement is
of 0\% ROC points. These results are of interest since these diffusion kernel
approaches are currently state-of-the-art approaches for gene-disease
prioritization.

\section{Conclusion and Future Work}
\label{conclusion}

% what is the contribution that is not trivial

In this paper, we have proposed the notion of {\em link enrichment} for
diffusion kernels, that is, the idea of carrying out the computation of
information diffusion on a network that contains edges identified by link
prediction approaches. We have discovered a robust signal that indicates that
a certain class of diffusion-based node kernels consistently benefits from the
coupling with link prediction techniques of the X type for a 

In future work we will carry out a more fine grained analysis, defining a
taxonomy of prediction problems on networks that make use of the notion of
node similarity and analyze which link prediction strategies can be
effectively coupled with specific node similarity computation techniques for
each given class of problems. In addition we will study the quantitative
relation between the degree of missingness and the size of the improvement
offered by prepending the link prediction to the node similarity assessment.
Finally we will extend the analysis to the more complex case of kernel
integration and data fusion, i.e. when multiple heterogeneous information
sources are used jointly to define the predictive task.